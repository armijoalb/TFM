\chapter*{}
%\thispagestyle{empty}
%\cleardoublepage

%\thispagestyle{empty}





\cleardoublepage
\thispagestyle{empty}

\begin{center}
{\large\bfseries Algoritmos de Deep Learning para el tratamiento de series temporales de clasificación}\\
\end{center}
\begin{center}
Alberto Armijo Ruiz\\
\end{center}

%\vspace{0.7cm}
\noindent{\textbf{Palabras clave}: LSTM, Deep Learning, Ciencia de Datos, Series Temporales, Clasificación, Desbalanceo de clases, Oversampling.\\

\vspace{0.7cm}
\noindent{\textbf{Resumen}}\\

Actualmente la Ciencia de Datos es una de las áreas de estudio que más repercusión está causando, dentro de esta, el Deep Learning es uno de los campos más famosos. Por ello el objetivo de este trabajo será estudiar la aplicación del Deep Learning a las Series Temporales de Clasificación con desbalanceo de clases.\newline

Para ello, se hará uso de las LSTM, uno de los tipos de red neuronal más utilizados en la actualidad. Durante el desarrollo de este trabajo se estudiarán dichas redes, el problema propuesto, posibles soluciones y diferentes técnicas de procesamiento de datos.\newline

Finalmente, se aplicarán y analizarán el uso de técnbicas de oversampling y selección de características para mejorar el rendimiento de las LSTM y otros modelos en el problema anterior.
\cleardoublepage


\thispagestyle{empty}


\begin{center}
{\large\bfseries Deep Learning algorithms for time series classification}\\
\end{center}
\begin{center}
Alberto Armijo Ruiz \\
\end{center}

%\vspace{0.7cm}
\noindent{\textbf{Keywords}: LSTM, Deep Learning, Data Science, Time Series, Classification, Class Imbalance, Oversampling.}\\

\vspace{0.7cm}
\noindent{\textbf{Abstract}}\\

Currently Data Science is one of the areas of study that is causing more impact, within this, Deep Learning is one of the most famous fields. Therefore, the objective of this work will be to study the application of Deep Learning to the  Time Series Classification with class imbalance.\newline

For this purpose, LSTM will be used, one of the most widely used types of neural networks today. During the develpment of this work we will study these neural networks, the proposed problem, possible solutions and different data processing techniques.\newline

Finally, the use of oversampling and feature selection techniques to improve the performance of LSTMs and other models in the above problem will be applied and analyzed.


\chapter*{}
\thispagestyle{empty}

\noindent\rule[-1ex]{\textwidth}{2pt}\\[4.5ex]

D. \textbf{Salvador García López}, Profesor del Área de Soft Computing and Intelligent Information Systems del Departamento DECSAI de la Universidad de Granada.

\vspace{0.5cm}

\textbf{Informan:}

\vspace{0.5cm}

Que el presente trabajo, titulado \textit{\textbf{Algoritmos de Deep Learning para el tratamiento de series temporales de clasificación}},
ha sido realizado bajo su supervisión por \textbf{Alberto Armijo Ruiz}, y autorizamos la defensa de dicho trabajo ante el tribunal
que corresponda.

\vspace{0.5cm}

Y para que conste, expiden y firman el presente informe en Granada a 12 de Agosto de 2019 .

\vspace{1cm}

\textbf{Los directores:}

\vspace{5cm}

\noindent \textbf{Salvador García López}


