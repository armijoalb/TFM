\chapter{Introducción}
\section{Objetivos}
Los objetivos propuesto para este Trabajo Fin de Máster son los siguientes:

\begin{itemize}
	\item Estudio de algoritmos básicos para series temporales y clasificación, tratamiento de series temporales de clasificación.
	\item Estudio sobre problema de desbalanceo de clases, métodos para tratar con desbalanceo, centrándose en el estudio de técnicas de oversampling utilizadas actualmente, métricas utilizadas para este tipo de problema.
	\item Estudio sobre LSTM y aplicación de estas al problemas de clasificación.
	\item Realización de pruebas sobre un conjunto de datasets de series temporales de clasificación con desbalanceo de clases, comparación y análisis de los resultados obtenidos por los diferentes modelos.
	\item Obtener conclusiones y trabajos futuros.
\end{itemize}
\newpage
\section{Motivación}
A día de hoy, la minería de datos en series temporales son un tipo de problema en auge. La Clasificación de Series Temporales (TSC, Time Series Classification en Inglés) es una de las áreas donde más investigación se está realizando en los últimos años debido a su gran número de aplicaciones en diferentes sectores como la industria, salud o transporte \cite{cao2013integrated} \cite{cao2014parsimonious} \cite{he2017uncertainty} \cite{xu2018spatio} \cite{roychoudhury2017cost} \cite{liang2013effective} \cite{geng2018cost} . Obtener una buena precisión en este tipo de problema puede ofrecer grandes beneficios por lo que existe un gran interés en este área.\newline

La Clasificación de Series Temporales tiene como objetivo clasificar datos durante el tiempo basado en su comportamiento. Por ejemplo, clasificar el comportamiento de un servidor durante el tiempo puede ayudar a detectar anomalías y a que la resolución de cualquier problema asociado a una anomalía sea más rápido.\newline

El desbalanceo de clases es un problema asociado a la clasificación y muy común en problemas de la vida real. La mayoría de los algoritmos clásicos suelen trabajar sobre la suposición de que el número de datos de cada clase están balanceados, por lo que el desbalanceo entre las diferentes clases hace que el rendimiento de estos algoritmos sea pobre. Esto hace que cuando la clase minoritaria es aquella que tiene más interés en clasificar, como por ejemplo detección de un terremoto; se necesite buscar una solución.\newline

Existen diferentes soluciones para intentar abordar el problema de desbalanceo entre clases a diferentes niveles: a nivel de datos, a nivel del algoritmo o combinaciones de estas dos. Los métodos que abordan el problema a nivel de datos intentan restablecer el balance entre clases mediante el muestreo de datos, añadiendo datos de la clase minoritaria (oversampling), eliminando datos de la clase mayoritaria (undersampling) o una combinación de ambos. Los métodos que abordan el problema a nivel del algoritmo enfatizan en la clase minoritaria manipulando e incorporando diferentes parámetros como el peso de cada clase.\newline

Por otro lado, las redes neuronales son unos de los modelos más famosos actualmente \cite{chung2015gated} \cite{cho2014properties} \cite{karim2017lstm} \cite{swapna2018automated} y se utilizan en un gran número diferentes de problemas obteniendo buenos resultados. Las redes formadas por LSTM (Long Short Term Memory) son un tipo de red que se utiliza en un gran número de problemas de gran complejidad como es la toma de decisiones en videojuegos, reconocimiento del habla o traducción de textos. Este tipo de redes son especialmente buenas en problemas donde hay que procesar secuencias de datos, como  es el caso de las series temporales. Además, la redes neuronales son capaces de trabajar con datos con alta dimensionalidad obteniendo buenos resultados a diferencia de otros métodos de minería de datos clásicos.\newline

En este trabajo se utilizarán diferentes conjuntos de datos sobre Clasificación de Series Temporales. Los conjuntos de datos utilizados de clasificación binaria y contienen desbalanceo entre las clases. Para el procesamiento de dichos conjuntos de datos se utilizará una red de LSTM y diferentes métodos de procesamiento para intentar obtener un buen rendimiento.\newline

Para paliar el desbalanceo de clases se utilizarán métodos de oversampling, en concreto SMOTE; este algoritmo es uno de los más utilizados para oversampling y suele ofrecer buenos resultados. Los resultados obtenidos se compararán con algoritmos clásicos de Machine Learning como por ejemplo SVM.
\newpage

\section{Organización de la memoria}
La memoria de este TFM se organizará en ocho capítulos. En el primer capítulo se mostrarán los objetivos de este trabajo y se describirá la motivación para su realización.\newline

El segundo capítulo trata sobre antecedentes, métodos de clásificación clásicos y métodos de predicción de series temporales, teoría básica sobre funcionamiento de LSTMs y algunas aplicaciones, por último, teoría sobre el problema de desbalanceo de clases.\newline

El tercer, cuarto y quinto capítulo describen tanto métodos de preprocesamiento, funcionamiento de métodos de oversampling que se utilizarán en el estudio, características específicas al preprocesar series temporales de clasificación  como implementación de una red de LSTMs para clasificación.\newline

El sexto capítulo contiene la información sobre los datos que se utilizan en el estudio, la arquitectura de la red básica que se utilizará y medidas para evaluar el rendimiento de la red. El capítulo siete contiene los resultados de las pruebas realizadas y un análisis sobre estos. Por último, el capítulo ocho contiene las conclusiones obtenidas por el estudio.