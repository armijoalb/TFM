\chapter{Conclusiones y trabajos futuros}
En este capítulo se comentarán las conclusiones alcanzas después de realizar las pruebas y el análisis visto en el capítulo anterior. Tras esto, se propondrán algunas líneas de trabajo que podrían salir de este trabajo.\newline

\section{Conclusiones}
Según los resultados mostrados en el capítulo anterior, se puede afirmar que las \textit{LSTM} son un buen modelo para predicción con series temporales de clasificación, incluso si tienen desbalanceo, y competitivo con el resto de modelos de clasificación clásica.\newline

De entre los métodos de preprocesamiento utilizados, las técnicas de oversampling no han resultados ser demasiado útiles, posiblemente esto sea por el hecho de ser algoritmos generales que no están centrados en las características de las series temporales. Para el caso de los modelos de \textit{LSTM} suele haber sobreajuste hacia la clase positiva, probablemente provocado por la forma de entrenar de este tipo de modelo y la creación de instancias de los métodos de oversampling. Las \textit{LSTM} entrenan con una serie de ejemplos en cada época y reajusta sus parámetros según el error que ha producido, los ejemplos creados con los métodos de oversampling están al final del conjunto de datos, por lo que en las últimas épocas los modelos solamente entrenan con ejemplos sintéticos de la clase positiva y se pierde la información aprendida de los ejemplos de la clase negativa. \newline

El método de la ventana ha dado buenos resultados en la gran mayoría de casos donde el conjunto de datos era una serie temporal multivariada de clasificación con desbalanceo, o cuando la extracción de caraterísticas era la adecuada cuando se trataba de clasificación de series temporales; por lo tanto, este tipo de procesamiento es interesante utilizarlo cuando se obtienen buenos resultados de la extracción de características y cuando se trabaja con el primer tipo de serie temporal comentado.\newline

Por último, la selección de características en la mayoría de los casos ha sido la causa de obtención de modelos que clasifiquen de forma perfecta el conjunto de datos durante las pruebas, por lo que se puede concluir que la extracción de características debe ser un preprocesamiento a probar siempre y a estudiar con más profundidad para este tipo de problemas.\newline

\section{Trabajos Futuros}
Como trabajos futuros pueden realizarse las siguientes propuestas:
\begin{itemize}
	\item Desarrollo de un método de oversampling específico para series temporales que tenga en cuenta las características específicas de esta.
	\item Estudio de otras formas de combatir el desbalanceo e implementación de un optimizador de matriz de pesos con \textit{LSTMs}.
	\item Estudio del uso de la ventana en más profundidad, estudio de medidas de las redes que indiquen un cierto tamaño específico por ejemplo, algoritmo para seleccionar una ventana adecuada.
	\item Estudio de más métodos de selección de características aplicados a las series temporales de clasificación.
\end{itemize}